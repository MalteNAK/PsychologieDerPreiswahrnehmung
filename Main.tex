\documentclass[11pt,bibtotoc]{article}
\usepackage[table]{xcolor}
\usepackage[ngerman]{babel} % Für Silbentrennung
\usepackage[utf8]{inputenc} % Um Umlaute und Sonderzeichen nutzen zu können
\usepackage[a4paper,
lmargin={2.5cm}, % 2.5cm links
rmargin={2.5cm}, % 2.5cm rechts
tmargin={2.5cm},
bmargin={2.75cm}]{geometry} % Für das Layout
\usepackage{setspace} % Für den Zeilenabstand
\setstretch{1.25}
\usepackage{graphicx} % Um Bilder einzufügen
\usepackage{wrapfig} % Um Bilder im Fließtext einzubinden
\usepackage{csquotes} % Um Text in Anführungsstriche zu setzen
\usepackage{mathptmx} % Ähnlich zu Times New Roman
\usepackage{fancyhdr} % Um den Header einzufügen
\usepackage{pdfpages} % Um die Titelseite einbinden
\usepackage{hyperref} % Um Verlinkungen einfügen
\usepackage{acronym} % Um Akronyme zu definieren
\usepackage{textcomp} % Um spezielle Zeichen verwenden zu können (z.B. °, µ)
\usepackage{subcaption} % Um mehrere Bilder zusammen einzustellen und Links zu Abbildungen zu korrigieren
\usepackage{tabularx}
\usepackage{booktabs}
\usepackage{caption}
\usepackage{enumitem}
\usepackage{float}

\usepackage[
  backend=biber,
  style=authoryear,
  autocite=footnote,
  maxcitenames=2,
  mincitenames=1,
  maxbibnames=99
]{biblatex}
\addbibresource{Quellenverzeichnis.bib}

\DefineBibliographyStrings{ngerman}{andothers={et al.}}


% Zitierungen kursiv innerhalb runder Klammern setzen
% \renewcommand*{\mkbibparens}[1]{\mkbibemph{(#1)}}
% \let\cite\parencite

\setlength{\parindent}{0pt} % Verhindert das Einrücken des ersten Satzes jedes Absatzes

\newcommand{\note}[1]{%
  {\color{red} #1}%
}


\pagestyle{fancy}
\fancyhf{} % Leert den Kopf- und Fußbereich
\RequirePackage{soul}
% Kopfzeilenbreite auf den Textbereich anpassen
\setlength{\headheight}{42.0pt} % Setzt die Kopfzeilenhöhe
\setlength{\headsep}{25pt} % Setzt den Abstand zwischen der Linie und dem Textkörper
\renewcommand{\headrulewidth}{2pt} % Stellt die Linienbreite auf 2pt
\setlength{\headwidth}{\dimexpr\textwidth} % Setzt die Linie so weit wie der Textbereich geht

% Kapitelname links und Logo rechts
\fancyhead[L]{\nouppercase{\leftmark}} % Zeigt das Kapitel links an
\fancyhead[R]{\includegraphics[width=5.5cm]{Resources/Nordakademie_Logo.png}} % Zeigt das Logo rechts an

\cfoot{\thepage} % Fußzeile mit Seitenzahl

\newcounter{romanBeginningEnd} % Um Römische Seitenzahl zu speichern

% Sektionen referenzen anpassen
\addto\extrasngerman{%
	\renewcommand{\sectionautorefname}{siehe}%
	\let\subsectionautorefname\sectionautorefname%
	\let\subsubsectionautorefname\sectionautorefname%
}

\begin{document}
    \pagenumbering{gobble}
    \begin{titlepage}
    \begin{center}

        \includegraphics[width=0.9\textwidth]{Resources/Nordakademie_Logo.png}
        
        \vspace{1cm}
        
        \huge
        \textbf{Psychologie der Preiswahrnehmung\\
        \textit{\enquote{Wie Unternehmen psychologische Effekte für ihre Preisgestaltung nutzen}}\\
        }
        \vspace{1.5cm}
        \huge
        \textbf{Texthandout}\\
        WPF: Marktpsychologie und Online Marketing    
        \vspace{1.5cm}

        \Large
        vorgelegt von:\\
        \vspace{0.5cm} 
        \textbf{Lennart Hell}\\
        Matr. Nr.: ????? \\
        \Large
        und\\
        \textbf{Helene Günther}\\
        Matr. Nr.: 13151\\
        \Large
        und\\
        \textbf{Malte Wendeler}\\
        Matr. Nr.: 13105
        
            
        \vspace{2.5cm}

    \end{center}
        \Large
        \textbf{Gutachter:} Prof.\ Dr.\ Thomas Gey
            
\end{titlepage}

	\setcounter{page}{1} % Römische Seitenzahlen
	\pagenumbering{Roman}
	
	\tableofcontents % Inhaltsverzeichnis
	\newpage
	
	\setcounter{secnumdepth}{0} % Keine Numerierung von Überschriften
	\clearpage
	\phantomsection

	\section{Abkürzungsverzeichnis}
	% Abkürzungen
	\begin{acronym}%[KRITIS] % Für Formatierung längste Abkürzung eintragen
	\acro{BEISPIEL}[BEISPIEL]{Beispiel Definition}
	\end{acronym}
    
	\newpage
	
	\setcounter{secnumdepth}{3} % Überschriften wieder numerieren
	
	\setcounter{romanBeginningEnd}{\the\value{page}} % Speichern der römischen Seitenzahl
	\setcounter{page}{1} % Mit Arabischen Seitenzahlen wieder bei 1 anfangen
	\pagenumbering{arabic}
	
	% ###############################################################
	% Hier Inhalt einfügen
	\section{Grundlagen der Preiswahrnehmung}
\subsection{Marketing als Bedürfnisbefriedigung: Der Weg zur Preisentscheidung}

(Einstieg auf der Konsumentenebene)\\

Ziel: Verständnis dafür schaffen, dass Preisentscheidungen nicht am Preisschild beginnen, sondern beim Bedürfnis.\\

Inhaltliche Leitlinie:\\

Marketing nicht als Absatztechnik, sondern als systematische Bedürfnisbefriedigung\\

Kaufentscheidung als Prozess, nicht als Einzelakt\\

Strukturidee:\\

Bedürfnisentstehung (physiologisch / sozial / psychologisch)\\

Emotionale Aktivierung als Bindeglied\\

Kaufentscheidung als emotional geprägter Akt\\

Preis als Trigger innerhalb dieses Prozesses, nicht als objektiver Maßstab\\

Brücke zu später:\\
Hier bereitest du vor, warum Effekte wie Zero-Price, Anchoring oder Prospect Theory überhaupt wirken können.

\subsection{Vom Homo Oeconomicus zum emotionalen Entscheider}
(Paradigmenwechsel im Marketingdenken)\\

Ziel: Aufräumen mit dem Mythos rationaler Konsumenten.\\

Inhaltliche Leitlinie:\\

Klassische ökonomische Sicht: rational, nutzenmaximierend\\

Empirische Widerlegung durch Verhaltensökonomie \& Konsumentenpsychologie.\\

Strukturidee:\\

Homo Oeconomicus als theoretisches Ideal\\

Grenzen rationaler Modelle (Informationsüberlastung, Heuristiken)\\

Emotionen, Intuition, Erfahrung als dominante Entscheidungsfaktoren\\

Relevanz für Preiswahrnehmung: Preis nicht gleich objektive Zahl\\

Wissenschaftliche Anknüpfung:\\

Kahneman \& Tversky (Prospect Theory)\\

Behavioral Economics als Fundament für Kapitel 2\\

\subsection{Entwicklung des Marketing-Mix: Von statisch zu dynamisch}

(Einordnung der 4 P's mit Fokus auf Wandel)\\

Ziel: Zeigen, dass sich Marketinginstrumente mit dem Menschen verändert haben.\\

Strukturidee:\\

Klassischer Marketing-Mix (Product, Price, Place, Promotion)\\

Ursprünglich: produkt- und kostenorientiert\\

Heute: konsumenten-, kontext- und erlebnisorientiert

Kurzer Überblick:

Product: Nutzen --> Erlebnis --> Bedeutung

Place: Distribution --> Verfügbarkeit --> Omnichannel

Promotion: Information --> Emotion --> Interaktion

Price: von Kostenfaktor --> Kommunikationsinstrument

Wichtig: Preis hier noch nicht vertiefen – das kommt in 1.4.

\subsection{Der Preis im Wandel: Vom objektiven Wert zum psychologischen Signal}

(Kernkapitel deines Themas)

Ziel: Preis als emotionales, kommunikatives Signal etablieren.

Inhaltliche Leitlinie:

Preis informiert nicht nur über Kosten, sondern über:

Qualität

Fairness

Status

Exklusivität

Preis wird interpretiert, nicht berechnet

Strukturidee:

Preis als Anker, Referenz und Vergleichspunkt

Subjektive Preisfairness

Kontextabhängigkeit von Preiswahrnehmung

Warum „teuer“ und „günstig“ psychologische Konstrukte sind

Direkte Vorbereitung auf Kapitel 2:
Hier kannst du explizit ankündigen, dass nun konkrete psychologische Effekte folgen (Zero Price, Anker, Veblen etc.).

\subsection{Globalisierung und Digitalisierung als Verstärker der Preispsychologie}

(Makroebene, die die Relevanz deines Themas rechtfertigt)

Ziel: Zeigen, warum Preiswahrnehmung heute wichtiger denn je ist.

Strukturidee:

Globalisierung --> höhere Vergleichbarkeit, Preistransparenz

Digitalisierung --> permanente Verfügbarkeit von Preisen

Algorithmen, dynamische Preise, personalisierte Angebote

Informationsüberfluss verstärkt emotionale Vereinfachung

Zentrale These:

Je rationaler der Markt wirkt, desto emotionaler entscheidet der Mensch.

Übergang:
Diese Bedingungen erklären, warum psychologische Effekte im Preis so mächtig geworden sind.

\subsection{Zwischenfazit: Preiswahrnehmung als emotionaler Entscheidungsprozess}

(Explizite Hinführung zu Kapitel 2)

Ziel: Theoretische Klammer + Ausblick.

Inhaltliche Leitlinie:

Preisentscheidungen sind:

subjektiv

kontextabhängig

emotional geprägt

Klassische Preis-Nachfrage-Modelle greifen zu kurz

Notwendigkeit psychologischer Erklärungsansätze

Direkter Übergang:
„Aufbauend auf diesen Grundlagen werden im folgenden Kapitel zentrale psychologische Mechanismen der Preiswahrnehmung analysiert.“
	\section{Psychologische Mechanismen der Preiswahrnehmung (Helene)}

\subsection{Zero-Price-Effekt}
Der Zero-Price-Effekt beschreibt ein Phänomen, bei dem Menschen häufig irrational handeln, sobald etwas kostenlos angeboten wird.\autocite{shampanier2007}
Dieses Verhalten unterscheidet sich deutlich von der Reaktion auf Angebote auch wenn diese nur sehr geringe Preise haben. Ein kostenloses Angebot wird dabei oft systematisch überbewertet.\autocite{shampanier2007}
Der Grund hierfür liegt darin, dass es im Kopf des Konsumenten keinerlei „Kosten“ verursacht, weder finanzieller Art noch in Form von Entscheidungsaufwand.\autocite{shampanier2007}
Dies führt zu einer radikalen Vereinfachung der Kaufentscheidung, da das komplexe Abwägen, ob der Nutzen den Preis rechtfertigt, komplett entfällt.\autocite{shampanier2007}

Hinter diesem Verhalten verbergen sich spezifische psychologische Prozesse. Zum einen stellen Menschen an Gratis-Produkte oft niedrigere Erwartungen; werden diese dann übertroffen, führt dies zu einer höheren Zufriedenheit als bei bezahlten Produkten.\autocite{shampanier2007}
Zum anderen entfällt durch den Wegfall der Kosten-Nutzen-Abwägung eine signifikante kognitive Hürde, wodurch die Entscheidung deutlich leichter fällt.\autocite{shampanier2007}
Hinzu kommt die „magnetische Kraft“ des Wortes „Gratis“ (Free), die einen besonderen Reiz ausübt und Konsumenten dazu verleitet, Dinge auszuwählen oder zu nutzen, die sie bei einem noch so niedrigen Preis ignoriert hätten.\autocite{shampanier2007}
In der Wirtschaft wird dieser Effekt vielfältig genutzt (z.\,B. Freemium-Modelle oder Mindestbestellwerte für kostenlosen Versand).\autocite{baier2024}

Der Erfolg dieser Strategie ist jedoch nicht garantiert. Nutzer können sich an Gratis-Angebote gewöhnen, was es später erschwert, Geld für die Leistung zu verlangen.\autocite{baier2024}
Zudem kann übertriebene Großzügigkeit Misstrauen wecken und die Frage nach dem „Haken“ aufwerfen, was dem Markenimage schadet.\autocite{baier2024}
Neuere Forschung weist zudem auf einen sogenannten „Boomerang-Effekt“ hin: Wenn mit dem Gratis-Angebot hohe Nebenkosten wie lange Wartezeiten oder großer Aufwand verbunden sind, kann die Nachfrage sogar sinken.\autocite{baier2024}
In solchen Fällen wirkt ein geringer Preis oft seriöser und attraktiver als ein Nullpreis.\autocite{baier2024}

\subsection{Preisschwelleneffekt (Price Threshold Effect)}
Entgegen der klassischen ökonomischen Annahme, die von einer stetigen Beziehung zwischen Preis und Nachfrage ausgeht, nehmen Konsumenten Preise in der Realität nicht kontinuierlich wahr.\autocite{gedenk1999,pauwels2007}
Vielmehr verarbeiten sie Preisinformationen diskontinuierlich in Form von Sprüngen beziehungsweise Schwellen.\autocite{gedenk1999,pauwels2007}
Dies führt zur Entstehung von sogenannten Indifferenzbereichen: Solange sich kleine Preisänderungen innerhalb eines solchen Intervalls bewegen, werden sie vom Kunden kaum registriert und haben somit keinen nennenswerten Einfluss auf die Kaufentscheidung.\autocite{gedenk1999}
Die Nachfrage reagiert in diesen Bereichen starr.\autocite{pauwels2007}
Eine signifikante, oft schlagartige Änderung in der Preiswahrnehmung – und damit im Kaufverhalten – tritt erst dann ein, wenn der Preis eine bestimmte psychologische Grenze überschreitet.\autocite{gedenk1999,pauwels2007}

Ein klassisches Beispiel hierfür ist der kritische Sprung von 9,99\,€ auf 10,00\,€.\autocite{gedenk1999}
Diese spezifischen Grenzen werden als Preisschwellen bezeichnet.\autocite{pauwels2007}
Hinter diesem Phänomen steht eine kognitive Vereinfachungsstrategie des Gehirns: Um die Informationsverarbeitung zu erleichtern, neigen Konsumenten dazu, Preise mental abzurunden und sie in vereinfachte Kategorien einzuordnen (z.\,B. „unter 10 Euro“).\autocite{gedenk1999,baumgartner2007}
Diese kognitive Entlastung führt dazu, dass die exakte Ziffernfolge der Nachkommastellen oft ausgeblendet wird und weniger relevant ist als die Einordnung in das übergeordnete Preissegment.\autocite{baumgartner2007}

\subsection{Ankereffekt}
Der Ankereffekt (Anchoring) bezeichnet eine fundamentale kognitive Verzerrung, bei der Menschen bei Entscheidungen übermäßig stark auf die erste angebotene Information vertrauen – den sogenannten „Anker“.\autocite{mussweiler2000}
Dieser initial genannte Wert brennt sich im Gedächtnis ein und dient fortan als unbewusster Referenzpunkt, an dem alle nachfolgenden Informationen gemessen und eingeordnet werden.\autocite{mussweiler2000}
Selbst wenn dieser Anker für den tatsächlichen Wert eines Produkts sachlich völlig irrelevant oder willkürlich gewählt ist, beeinflusst er die Zahlungsbereitschaft des Konsumenten maßgeblich.\autocite{mussweiler2000}
Die mentale Anpassung vom Anker weg gelingt dem menschlichen Gehirn oft nur unzureichend.\autocite{mussweiler2000}

Im modernen Preismanagement wird dieser Effekt systematisch genutzt, um Preiswahrnehmungen gezielt zu steuern.\autocite{cialdini2008}
Das prominenteste Instrument ist der Einsatz von Referenzpreisen: Ein höherer, oft durchgestrichener Preis (z.\,B. eine unverbindliche Preisempfehlung) fungiert als Anker, der den aktuellen Verkaufspreis optisch verkleinert und die Kaufwahrscheinlichkeit erhöht.\autocite{cialdini2008}
Ähnliches gilt für die Sortimentsgestaltung durch extrem teure „Ankerprodukte“, die selten gekauft werden, aber Vergleichsrahmen setzen.\autocite{cialdini2008}

\subsection{Prospect Theory}
Die von Kahneman und Tversky entwickelte Prospect Theory liefert eine fundierte Erklärung dafür, warum wirtschaftliche Entscheidungen oft nicht rein rational getroffen werden.\autocite[263--291]{kahneman1979}
Ihre zentrale Erkenntnis lautet, dass Menschen Gewinne und Verluste keineswegs symmetrisch bewerten: Der psychologische Schmerz eines Verlusts wiegt stärker als die Freude über einen gleich großen Gewinn.\autocite[263--291]{kahneman1979}
Ein entscheidender Aspekt dabei ist die Referenzabhängigkeit: Konsumenten beurteilen Preise nicht isoliert, sondern relativ zu einem subjektiven Referenzpunkt.\autocite[263--291]{kahneman1979}
Ein Preis unterhalb dieses Referenzpunkts wird als Gewinn erlebt, ein Preis oberhalb als Verlust.\autocite[263--291]{kahneman1979}

Eng verknüpft mit diesen Mechanismen ist das Empfinden von „Preisfairness“. Preiserhöhungen ohne nachvollziehbaren Grund werden vor dem Hintergrund der Verlustaversion schneller als unfair wahrgenommen.\autocite[263--291]{kahneman1979}

\subsection{Veblen-Effekt}
Der Veblen-Effekt beschreibt eine ökonomische Anomalie, bei der die Nachfrage nach einem Produkt zunimmt, gerade weil dessen Preis hoch ist.\autocite{veblen1899}
Der hohe Preis stiftet einen Zusatznutzen, indem er soziale Exklusivität vermittelt und als Statussignal wirkt.\autocite{veblen1899}
Unternehmen nutzen dies strategisch im Rahmen von Prestige Pricing.\autocite{veblen1899}

\subsection{Snob-Effekt}
Der Snob-Effekt ist mit dem Veblen-Effekt verwandt, unterscheidet sich jedoch darin, dass der Reiz nicht primär vom hohen Preis, sondern von der Exklusivität durch Seltenheit und Abgrenzung gegenüber dem Mainstream ausgeht.\autocite{veblen1899}
Der Preis wirkt hier eher indirekt als Barriere, um Verbreitung zu begrenzen.\autocite{veblen1899}
Unternehmen können diesen Effekt durch künstliche Verknappung (Limited Editions, selektive Distribution) verstärken.\autocite{veblen1899}

	% ###############################################################
	
	\newpage
	\pagenumbering{Roman}
	\setcounter{page}{\theromanBeginningEnd} % Römische Seitenzahlen fortsetzen
	\setcounter{secnumdepth}{0} % Nummerierung der Überschriften entfernt
	\section{Quellenverzeichnis}
	\setcounter{secnumdepth}{3} % Überschriften wieder numerieren (Für den Anhang)
	\printbibliography[heading=none]
	\clearpage
	\setcounter{secnumdepth}{0} % Nummerierung der Überschriften entfernt
	\section{Eigenständigkeitserklärung}
	\clearpage
	\section{Aufteilung der Inhalte}

\begin{table}[H]
\begin{tabular}{|l|l|}
\hline
Kapitel & Verfasser \\
\hline
% Tragen Sie hier Ihre Aufteilung ein
\hline
\end{tabular}
\end{table}

\end{document}
