\section{Grundlagen der Preiswahrnehmung}
\subsection{Marketing als Bedürfnisbefriedigung: Der Weg zur Preisentscheidung}

(Einstieg auf der Konsumentenebene)\\

Ziel: Verständnis dafür schaffen, dass Preisentscheidungen nicht am Preisschild beginnen, sondern beim Bedürfnis.\\

Inhaltliche Leitlinie:\\

Marketing nicht als Absatztechnik, sondern als systematische Bedürfnisbefriedigung\\

Kaufentscheidung als Prozess, nicht als Einzelakt\\

Strukturidee:\\

Bedürfnisentstehung (physiologisch / sozial / psychologisch)\\

Emotionale Aktivierung als Bindeglied\\

Kaufentscheidung als emotional geprägter Akt\\

Preis als Trigger innerhalb dieses Prozesses, nicht als objektiver Maßstab\\

Brücke zu später:\\
Hier bereitest du vor, warum Effekte wie Zero-Price, Anchoring oder Prospect Theory überhaupt wirken können.

\subsection{Vom Homo Oeconomicus zum emotionalen Entscheider}
(Paradigmenwechsel im Marketingdenken)\\

Ziel: Aufräumen mit dem Mythos rationaler Konsumenten.\\

Inhaltliche Leitlinie:\\

Klassische ökonomische Sicht: rational, nutzenmaximierend\\

Empirische Widerlegung durch Verhaltensökonomie \& Konsumentenpsychologie.\\

Strukturidee:\\

Homo Oeconomicus als theoretisches Ideal\\

Grenzen rationaler Modelle (Informationsüberlastung, Heuristiken)\\

Emotionen, Intuition, Erfahrung als dominante Entscheidungsfaktoren\\

Relevanz für Preiswahrnehmung: Preis nicht gleich objektive Zahl\\

Wissenschaftliche Anknüpfung:\\

Kahneman \& Tversky (Prospect Theory)\\

Behavioral Economics als Fundament für Kapitel 2\\

\subsection{Entwicklung des Marketing-Mix: Von statisch zu dynamisch}

(Einordnung der 4 P's mit Fokus auf Wandel)\\

Ziel: Zeigen, dass sich Marketinginstrumente mit dem Menschen verändert haben.\\

Strukturidee:\\

Klassischer Marketing-Mix (Product, Price, Place, Promotion)\\

Ursprünglich: produkt- und kostenorientiert\\

Heute: konsumenten-, kontext- und erlebnisorientiert

Kurzer Überblick:

Product: Nutzen --> Erlebnis --> Bedeutung

Place: Distribution --> Verfügbarkeit --> Omnichannel

Promotion: Information --> Emotion --> Interaktion

Price: von Kostenfaktor --> Kommunikationsinstrument

Wichtig: Preis hier noch nicht vertiefen – das kommt in 1.4.

\subsection{Der Preis im Wandel: Vom objektiven Wert zum psychologischen Signal}

(Kernkapitel deines Themas)

Ziel: Preis als emotionales, kommunikatives Signal etablieren.

Inhaltliche Leitlinie:

Preis informiert nicht nur über Kosten, sondern über:

Qualität

Fairness

Status

Exklusivität

Preis wird interpretiert, nicht berechnet

Strukturidee:

Preis als Anker, Referenz und Vergleichspunkt

Subjektive Preisfairness

Kontextabhängigkeit von Preiswahrnehmung

Warum „teuer“ und „günstig“ psychologische Konstrukte sind

Direkte Vorbereitung auf Kapitel 2:
Hier kannst du explizit ankündigen, dass nun konkrete psychologische Effekte folgen (Zero Price, Anker, Veblen etc.).

\subsection{Globalisierung und Digitalisierung als Verstärker der Preispsychologie}

(Makroebene, die die Relevanz deines Themas rechtfertigt)

Ziel: Zeigen, warum Preiswahrnehmung heute wichtiger denn je ist.

Strukturidee:

Globalisierung --> höhere Vergleichbarkeit, Preistransparenz

Digitalisierung --> permanente Verfügbarkeit von Preisen

Algorithmen, dynamische Preise, personalisierte Angebote

Informationsüberfluss verstärkt emotionale Vereinfachung

Zentrale These:

Je rationaler der Markt wirkt, desto emotionaler entscheidet der Mensch.

Übergang:
Diese Bedingungen erklären, warum psychologische Effekte im Preis so mächtig geworden sind.

\subsection{Zwischenfazit: Preiswahrnehmung als emotionaler Entscheidungsprozess}

(Explizite Hinführung zu Kapitel 2)

Ziel: Theoretische Klammer + Ausblick.

Inhaltliche Leitlinie:

Preisentscheidungen sind:

subjektiv

kontextabhängig

emotional geprägt

Klassische Preis-Nachfrage-Modelle greifen zu kurz

Notwendigkeit psychologischer Erklärungsansätze

Direkter Übergang:
„Aufbauend auf diesen Grundlagen werden im folgenden Kapitel zentrale psychologische Mechanismen der Preiswahrnehmung analysiert.“